\documentclass{article}

% Language setting
% Replace `english' with e.g. `spanish' to change the document language
\usepackage[english]{babel}

% Set page size and margins
% Replace `letterpaper' with`a4paper' for UK/EU standard size
\usepackage[letterpaper,top=2cm,bottom=2cm,left=3cm,right=3cm,marginparwidth=1.75cm]{geometry}

% Useful packages
\usepackage{amsmath}
\usepackage{graphicx}
\usepackage[colorlinks=true, allcolors=blue]{hyperref}

\title{Extraordinario de Pruebas de Software}
\author{Itzel Rincon de la Cruz}

\begin{document}
\maketitle
\section{Introduccion}

En este extraordinario de pruebas de software, implementaremos los conocimientos adquiridos a travez del curso como son:
- Uso de Angular
- Pruebas unitarias
- Conocimientos del lenguaje TypeScript
- Implementacion de firebase
- Manejo de proyectos automatizados con GitHub

\section{Tema:}
\subsection{Ley de la gravedad - Newton}
La ley de gravitación universal es una ley física clásica que describe la interacción gravitatoria entre distintos cuerpos con masa. Fue formulada por Isaac Newton en su libro Philosophiae Naturalis Principia Mathematica, publicado el 5 de julio de 1687, donde establece por primera vez una relación proporcional (deducida empíricamente de la observación) de la fuerza con que se atraen dos objetos con masa. Así, Newton dedujo que la fuerza con que se atraen dos cuerpos tenía que ser proporcional al producto de sus masas dividido por la distancia entre ellos al cuadrado. Para grandes distancias de separación entre cuerpos se observa que dicha fuerza actúa de manera muy aproximada como si toda la masa de cada uno de los cuerpos estuviese concentrada únicamente en su centro de gravedad, es decir, es como si dichos objetos fuesen únicamente un punto, lo cual permite reducir enormemente la complejidad de las interacciones entre cuerpos complejos.
\end{document}